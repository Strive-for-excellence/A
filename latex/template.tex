\documentclass{article}
\usepackage[margin=1in]{geometry}
\usepackage[Glenn]{fncychap}
\usepackage{listings}
\usepackage{color}
\usepackage{verbatim}
\usepackage{zh_CN-Adobefonts_external} 
\usepackage{fancyhdr}  % 页眉页脚
\usepackage{minted}    % 代码高亮
\usepackage[colorlinks]{hyperref}  % 目录可跳转

\title{Code Template for ACM-ICPC}
\author{zzuzxy}

\definecolor{dkgreen}{rgb}{0,0.6,0}
\definecolor{gray}{rgb}{0.5,0.5,0.5}
\definecolor{mauve}{rgb}{0.58,0,0.82}

\lstset{frame=tb,
  language=c++,
  aboveskip=3mm,
  belowskip=3mm,
  showstringspaces=false,
  columns=flexible,
  basicstyle={\small\ttfamily},
  numbers=none,
  numberstyle=\tiny\color{gray},
  keywordstyle=\color{blue},
  commentstyle=\color{dkgreen},
  stringstyle=\color{mauve},
  breaklines=true,
  breakatwhitespace=true
  tabsize=3
}

\begin{document}
\begin{titlepage}
\maketitle
\thispagestyle{empty}
\pagebreak
\pagestyle{plain}
\tableofcontents
\end{titlepage}
\section{1 数据结构}
\subsection{二叉树.cpp}
\inputminted{c++}{/home/zzuzxy/t2/Template1.0/1 数据结构/二叉搜索树/二叉树.cpp}
\subsection{堆.cpp}
\inputminted{c++}{/home/zzuzxy/t2/Template1.0/1 数据结构/基础数据结构/堆.cpp}
\subsection{字符串}
\subsubsection{1 Trie(前缀树).cpp}
\inputminted{c++}{/home/zzuzxy/t2/Template1.0/1 数据结构/字符串/1 Trie(前缀树).cpp}
\subsubsection{2 KMP.cpp}
\inputminted{c++}{/home/zzuzxy/t2/Template1.0/1 数据结构/字符串/2 KMP.cpp}
\subsubsection{3 AC自动机.cpp}
\inputminted{c++}{/home/zzuzxy/t2/Template1.0/1 数据结构/字符串/3 AC自动机.cpp}
\subsubsection{4 KMP-KMP变形.cpp}
\inputminted{c++}{/home/zzuzxy/t2/Template1.0/1 数据结构/字符串/4 KMP-KMP变形.cpp}
\subsubsection{5 字符串hash.cpp}
\inputminted{c++}{/home/zzuzxy/t2/Template1.0/1 数据结构/字符串/5 字符串hash.cpp}
\subsubsection{6 后缀数组.cpp}
\inputminted{c++}{/home/zzuzxy/t2/Template1.0/1 数据结构/字符串/6 后缀数组.cpp}
\section{2 动态规划}
\subsection{1 最长上升子序列.cpp}
\inputminted{c++}{/home/zzuzxy/t2/Template1.0/2 动态规划/1 最长上升子序列.cpp}
\section{3 图论}
\subsection{DFS}
\subsubsection{1.无向图的割点和桥.cpp}
\inputminted{c++}{/home/zzuzxy/t2/Template1.0/3 图论/DFS/1.无向图的割点和桥.cpp}
\subsubsection{2.无向图的双连通分量.cpp}
\inputminted{c++}{/home/zzuzxy/t2/Template1.0/3 图论/DFS/2.无向图的双连通分量.cpp}
\subsubsection{3有向图的强联通分量.cpp}
\inputminted{c++}{/home/zzuzxy/t2/Template1.0/3 图论/DFS/3有向图的强联通分量.cpp}
\subsubsection{4 2-sat 问题.cpp}
\inputminted{c++}{/home/zzuzxy/t2/Template1.0/3 图论/DFS/4 2-sat 问题.cpp}
\subsection{LCA}
\subsubsection{1 DFS+RMQ.cpp}
\inputminted{c++}{/home/zzuzxy/t2/Template1.0/3 图论/LCA/1 DFS+RMQ.cpp}
\subsubsection{2倍增算法.cpp}
\inputminted{c++}{/home/zzuzxy/t2/Template1.0/3 图论/LCA/2倍增算法.cpp}
\subsection{Maxflow}
\subsubsection{1 Dinic.cpp}
\inputminted{c++}{/home/zzuzxy/t2/Template1.0/3 图论/Maxflow/1 Dinic.cpp}
\subsubsection{2 ISAP.cpp}
\inputminted{c++}{/home/zzuzxy/t2/Template1.0/3 图论/Maxflow/2 ISAP.cpp}
\subsubsection{3 MCMF.cpp}
\inputminted{c++}{/home/zzuzxy/t2/Template1.0/3 图论/Maxflow/3 MCMF.cpp}
\subsection{二分图}
\subsubsection{KM.cpp}
\inputminted{c++}{/home/zzuzxy/t2/Template1.0/3 图论/二分图/KM.cpp}
\subsubsection{匈牙利算法.cpp}
\inputminted{c++}{/home/zzuzxy/t2/Template1.0/3 图论/二分图/匈牙利算法.cpp}
\subsection{最小生成树}
\subsubsection{1 Krustral 卡鲁斯卡尔算法.cpp}
\inputminted{c++}{/home/zzuzxy/t2/Template1.0/3 图论/最小生成树/1 Krustral 卡鲁斯卡尔算法.cpp}
\subsubsection{2 prim 算法.cpp}
\inputminted{c++}{/home/zzuzxy/t2/Template1.0/3 图论/最小生成树/2 prim 算法.cpp}
\subsubsection{3 最小限制生成树.cpp}
\inputminted{c++}{/home/zzuzxy/t2/Template1.0/3 图论/最小生成树/3 最小限制生成树.cpp}
\subsubsection{4 次小生成树.cpp}
\inputminted{c++}{/home/zzuzxy/t2/Template1.0/3 图论/最小生成树/4 次小生成树.cpp}
\subsection{最短路}
\subsubsection{1 Dijkstra.cpp}
\inputminted{c++}{/home/zzuzxy/t2/Template1.0/3 图论/最短路/1 Dijkstra.cpp}
\subsubsection{2 Bellman-ford.cpp}
\inputminted{c++}{/home/zzuzxy/t2/Template1.0/3 图论/最短路/2 Bellman-ford.cpp}
\subsubsection{3 floyed.cpp}
\inputminted{c++}{/home/zzuzxy/t2/Template1.0/3 图论/最短路/3 floyed.cpp}
\section{4 数学}
\subsection{3 FWT模板.cpp}
\inputminted{c++}{/home/zzuzxy/t2/Template1.0/4 数学/3 FWT模板.cpp}
\subsection{FFT}
\subsubsection{FFT.cpp}
\inputminted{c++}{/home/zzuzxy/t2/Template1.0/4 数学/FFT/FFT.cpp}
\subsubsection{kuangbin.cpp}
\inputminted{c++}{/home/zzuzxy/t2/Template1.0/4 数学/FFT/kuangbin.cpp}
\subsubsection{lrj.cpp}
\inputminted{c++}{/home/zzuzxy/t2/Template1.0/4 数学/FFT/lrj.cpp}
\subsection{template.cpp}
\inputminted{c++}{/home/zzuzxy/t2/Template1.0/4 数学/Lagrange-poly/template.cpp}
\subsection{数论}
\subsubsection{1 加法.cpp}
\inputminted{c++}{/home/zzuzxy/t2/Template1.0/4 数学/数论/1 加法.cpp}
\subsubsection{1 逆元.cpp}
\inputminted{c++}{/home/zzuzxy/t2/Template1.0/4 数学/数论/1 逆元.cpp}
\subsubsection{2 减法.cpp}
\inputminted{c++}{/home/zzuzxy/t2/Template1.0/4 数学/数论/2 减法.cpp}
\subsubsection{3 乘法.cpp}
\inputminted{c++}{/home/zzuzxy/t2/Template1.0/4 数学/数论/3 乘法.cpp}
\subsubsection{4 除法.cpp}
\inputminted{c++}{/home/zzuzxy/t2/Template1.0/4 数学/数论/4 除法.cpp}
\subsubsection{Euler.cpp}
\inputminted{c++}{/home/zzuzxy/t2/Template1.0/4 数学/数论/Euler.cpp}
\subsubsection{lucas ,组合数.cpp}
\inputminted{c++}{/home/zzuzxy/t2/Template1.0/4 数学/数论/lucas ,组合数.cpp}
\subsubsection{miller-rabin-Pollard-rho.cpp}
\inputminted{c++}{/home/zzuzxy/t2/Template1.0/4 数学/数论/miller-rabin-Pollard-rho.cpp}
\subsubsection{快速数论变换.cpp}
\inputminted{c++}{/home/zzuzxy/t2/Template1.0/4 数学/数论/快速数论变换.cpp}
\subsubsection{欧拉筛和埃氏筛.cpp}
\inputminted{c++}{/home/zzuzxy/t2/Template1.0/4 数学/数论/欧拉筛和埃氏筛.cpp}
\subsubsection{素性检测.cpp}
\inputminted{c++}{/home/zzuzxy/t2/Template1.0/4 数学/数论/素性检测.cpp}
\subsubsection{素数筛.cpp}
\inputminted{c++}{/home/zzuzxy/t2/Template1.0/4 数学/数论/素数筛.cpp}
\subsubsection{逆元打表.cpp}
\inputminted{c++}{/home/zzuzxy/t2/Template1.0/4 数学/数论/逆元打表.cpp}
\subsection{矩阵快速幂.cpp}
\inputminted{c++}{/home/zzuzxy/t2/Template1.0/4 数学/矩阵快速幂.cpp}
\subsection{自适应辛普森积分.cpp}
\inputminted{c++}{/home/zzuzxy/t2/Template1.0/4 数学/自适应辛普森积分.cpp}
\section{5 几何}
\subsection{2D}
\subsubsection{PSLG.cpp}
\inputminted{c++}{/home/zzuzxy/t2/Template1.0/5 几何/2D/PSLG.cpp}
\subsubsection{二维几何模板.cpp}
\inputminted{c++}{/home/zzuzxy/t2/Template1.0/5 几何/2D/二维几何模板.cpp}
\subsubsection{二维凸包.cpp}
\inputminted{c++}{/home/zzuzxy/t2/Template1.0/5 几何/2D/二维凸包.cpp}
\subsubsection{判断点是否在多边形内.cpp}
\inputminted{c++}{/home/zzuzxy/t2/Template1.0/5 几何/2D/判断点是否在多边形内.cpp}
\subsubsection{圆与多边形相交的面积.cpp}
\inputminted{c++}{/home/zzuzxy/t2/Template1.0/5 几何/2D/圆与多边形相交的面积.cpp}
\subsubsection{求圆与直线的交点.cpp}
\inputminted{c++}{/home/zzuzxy/t2/Template1.0/5 几何/2D/求圆与直线的交点.cpp}
\subsection{3D}
\subsubsection{三维几何的基本操作.cpp}
\inputminted{c++}{/home/zzuzxy/t2/Template1.0/5 几何/3D/三维几何的基本操作.cpp}
\subsubsection{三维几何的模版.cpp}
\inputminted{c++}{/home/zzuzxy/t2/Template1.0/5 几何/3D/三维几何的模版.cpp}
\subsubsection{三维凸包.cpp}
\inputminted{c++}{/home/zzuzxy/t2/Template1.0/5 几何/3D/三维凸包.cpp}
\subsubsection{维度转换为三维坐标.cpp}
\inputminted{c++}{/home/zzuzxy/t2/Template1.0/5 几何/3D/维度转换为三维坐标.cpp}
\section{6 其它}
\subsection{IO}
\subsubsection{fread.cpp}
\inputminted{c++}{/home/zzuzxy/t2/Template1.0/6 其它/IO/fread.cpp}
\subsubsection{fread2.cpp}
\inputminted{c++}{/home/zzuzxy/t2/Template1.0/6 其它/IO/fread2.cpp}
\subsubsection{保留小数.cpp}
\inputminted{c++}{/home/zzuzxy/t2/Template1.0/6 其它/IO/保留小数.cpp}
\subsubsection{读取整数.cpp}
\inputminted{c++}{/home/zzuzxy/t2/Template1.0/6 其它/IO/读取整数.cpp}
\subsection{c++中处理2进制的一些函数.cpp}
\inputminted{c++}{/home/zzuzxy/t2/Template1.0/6 其它/c++中处理2进制的一些函数.cpp}
\subsection{测量程序的运行时间.cpp}
\inputminted{c++}{/home/zzuzxy/t2/Template1.0/6 其它/测量程序的运行时间.cpp}
\end{document}